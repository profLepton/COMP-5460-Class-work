\documentclass{article}
\usepackage{graphicx} % Required for inserting images
\usepackage{setspace}

\singlespacing

\title{Literature Review 1}
\author{Akshay Kolli}
\date{18 February 2023}

\begin{document}

\maketitle



\section{Title and Authors}



\textbf{Primary Paper:} Using Artificial Intelligence to Visualize the Impacts of Climate Change

\bigskip
\noindent\textbf{Authors:} Alexandra Luccioni, Victor Schmidt, Vahe Vardanyan, Yoshua Bengio

\bigskip
\noindent\textbf{Source:} IEEE Computer Graphics and Applications ( Volume: 41, Issue: 1, 01 Jan.-Feb. 2021)

\bigskip
\noindent\textbf{Secondary Paper:} Embodied experiences in immersive virtual environments: Effects on pro-environmental attitude and behavior

\bigskip

\noindent\textbf{Author:} S. J. Ahn

\bigskip
\noindent\textbf{Source:} Doctoral dissertation, Stanford Univ. (2011)

\section{Overview}

The primary paper delves into the realm of artificial intelligence (AI) and how it can be utilized to portray and demonstrate the potential impacts of climate change. On the other hand, the secondary paper examines the effects of immersive virtual environments on pro-environmental attitudes and behavior. Both these papers provide insightful analysis of the intersection of technology and climate change. The primary paper offers a distinctive approach to comprehending climate change impacts, while the secondary paper provides a potential solution to encourage pro-environmental behavior. The amalgamation of these two papers would provide a comprehensive outlook on how the utilization of technology can positively impact our environment

\section{Relevance}

Climate change is a complex and urgent challenge that poses a significant threat to the planet and its inhabitants. It is, therefore, crucial to comprehend the potential impacts of climate change and investigate effective strategies to mitigate its effects. The primary paper suggests the application of Artificial Intelligence (AI) to generate realistic visualizations of the possible consequences of climate change on different regions globally. This innovative approach could aid policymakers and the public in comprehending the severe implications of inaction, thus prompting action to address this critical issue.

The secondary paper complements the primary paper by suggesting an immersive virtual environment as a potential solution to foster pro-environmental behavior. Such a solution may prove instrumental in raising public awareness about climate change, as it could encourage individuals to take action by immersing them in a digital environment that simulates the impact of human activities on the environment. Thus, the paper highlights the intersection of technology and climate change, offering a unique perspective on how technology can play a crucial role in addressing this pressing issue.

\section{Description}
The secondary paper delves into the potential of immersive virtual environments in promoting pro-environmental attitudes and behavior. The author conducted a study to investigate the effectiveness of these environments. Participants were exposed to an immersive virtual environment designed to simulate the experience of walking through a forest and asked to complete tasks related to protecting the environment, such as turning off lights and recycling.

The results of the study showed that participants who were exposed to the immersive virtual environment were more likely to engage in pro-environmental behavior and had more positive attitudes towards the environment compared to those who were not exposed to the virtual environment. The author suggests that immersive virtual environments could be a powerful tool for promoting pro-environmental behavior and encouraging action on climate change.

The primary paper, on the other hand, discusses the use of AI to create realistic visualizations of the potential impacts of climate change on different regions of the world. The authors trained a machine learning model using data from the Coupled Model Intercomparison Project 5 (CMIP5) to simulate future climate scenarios, which they then used to generate visualizations showing how temperature, precipitation, sea level rise, and other factors could change in the coming decades and centuries.

The authors also developed a web-based platform called ClimateNet that enables users to interact with these visualizations and explore the potential impacts of climate change in various regions of the world. The platform includes features that allow users to customize their experience, view climate change impacts from different perspectives, and compare different climate scenarios.

Using AI to create climate change visualizations provides policymakers and the public with realistic, interactive visualizations that better help them understand the consequences of inaction and be more motivated to take action to mitigate climate change. AI can also generate more accurate and detailed climate change projections, which can help inform policy decisions and resource allocation.

In summary, both papers address the intersection of technology and climate change. The primary paper proposes using AI to create visualizations of climate change impacts to help policymakers and the public better understand the consequences of inaction. In contrast, the secondary paper explores the potential of immersive virtual environments in promoting pro-environmental behavior. Both papers offer unique and promising approaches to address the urgent challenge of climate change.

\section{Summary}

The primary paper proposes using artificial intelligence (AI) to create realistic visualizations of how climate change could impact different regions of the world. The authors trained a machine learning model to simulate future climate scenarios using data from the Coupled Model Intercomparison Project 5 (CMIP5) and used this data to generate realistic visualizations of climate change impacts on different regions of the world. The visualizations show how temperature, precipitation, sea level rise, and other factors could change in the coming decades and centuries.

The authors also developed a web-based platform called ClimateNet that allows users to interact with these visualizations and explore the potential impacts of climate change in different regions of the world. The platform includes a variety of tools and features that allow users to customize their experience and view climate change impacts from different perspectives.

The primary paper highlights the potential benefits of using AI to create climate change visualizations. By providing realistic, interactive visualizations of climate change impacts, policymakers and the public can better understand the consequences of inaction and be more motivated to take action to mitigate climate change. Additionally, AI can help generate more accurate and detailed climate change projections, which can help inform policy decisions and resource allocation.

\end{document}
